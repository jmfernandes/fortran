

\documentclass[12pt]{article}
\usepackage{graphicx}
\usepackage{amsmath}
\usepackage{listings}
\usepackage{color}
\usepackage[section]{placeins} %this stops the figures from showing up in wrong section

\definecolor{dkgreen}{rgb}{0,0.6,0}
\definecolor{dkblue}{rgb}{0,0.0,0.6}
\definecolor{dkred}{rgb}{0.9,0.0,0.1}


\begin{document}

\lstset{language=Fortran,tabsize=4,numbers=left,numberstyle=\tiny,basicstyle=\ttfamily\small\color{dkblue},stringstyle=\ttfamily\color{blue},keywordstyle=\rmfamily\color{dkred}\bfseries\emph,backgroundcolor=\color{white},commentstyle=\color{dkgreen}}




\title{Physics 562 - Computational Physics\\[.5cm]
Midterm 2}
\author{Josh Fernandes\\
Department of Physics \& Astronomy\\
California State University Long Beach}
\date{\today}

  
\maketitle



\begin{abstract}
This paper examines two different questions. 
\end{abstract}

\section{Problem 1}\label{s:intro}
The first problem is an interesting one. 


\section{The Fortran95 code}

Numtype is the same for problems 1 and 2. 
\begin{lstlisting}[frame=single,caption={Module {\tt NumType}},label=module]

module NumType

	save
	integer, parameter :: 	dp = kind(1.d0)	
	real(dp), parameter :: 	pi = 4*atan(1._dp)
	complex(dp), parameter :: 	iic = (0._dp,1._dp),&	
								one = (1._dp,0._dp),&
								zero = (0._dp,0._dp)


end module NumType

\end{lstlisting}

\begin{lstlisting}[frame=single,caption={ {\tt q1.f95}},label=module]



\end{lstlisting}

The main program is {\tt q1} and it begins with its own module. 

\begin{lstlisting}[frame=single,caption={{\tt q1.f95}},label=adpend]



\end{lstlisting}


The code is run by typing {\tt ./q1}. Various data sets are plotted to different files for easy graphing.

\section{Problem 2}
The second problem is challenging.


\section{The Fortran95 code}

\begin{lstlisting}[frame=single,caption={ {\tt q2.f95}},label=module]



\end{lstlisting}

The main program is {\tt q2} and it begins with its own module. 

\begin{lstlisting}[frame=single,caption={{\tt q2.f95}},label=adpend]



\end{lstlisting}


The code is run by typing {\tt ./q2}. Various data sets are plotted to different files for easy graphing.


\section{Results}

Both the results are talked about here. 

\begin {figure}[!htb]
	\includegraphics[width=1.\textwidth]{question_1/one.png}
	%\resizebox{\columnwidth}{!}{\input{question_1/one.png}}
	\caption{first image }
	\label{imageone}
\end {figure}

\begin {figure}[!htb]
	\includegraphics[width=1.\textwidth]{question_2/two.png}
	%\resizebox{\columnwidth}{!}{\input{question_2/two.png}}
	\caption{second image}
	\label{imagetwo}
\end {figure}




\section{Summary and conclusions}

Both problems have interesting results

\begin{thebibliography}{}


\bibitem{metcalf} M.\ Metcalf, J.\ Reid and M.\ Cohen, {\it Fortran 95/2003 explained}. Oxford University Press, 2004.
 

\end{thebibliography}




\end{document}
