

\documentclass[12pt]{article}
\usepackage{graphicx}
\usepackage{amsmath}
\usepackage{listings}
\usepackage{color}
\usepackage[section]{placeins} %this stops the figures from showing up in wrong section

\definecolor{dkgreen}{rgb}{0,0.6,0}
\definecolor{dkblue}{rgb}{0,0.0,0.6}
\definecolor{dkred}{rgb}{0.9,0.0,0.1}


\begin{document}

\lstset{language=Fortran,tabsize=4,numbers=left,numberstyle=\tiny,basicstyle=\ttfamily\small\color{dkblue},stringstyle=\ttfamily\color{blue},keywordstyle=\rmfamily\color{dkred}\bfseries\emph,backgroundcolor=\color{white},commentstyle=\color{dkgreen}}




\title{Physics 562 - Computational Physics\\[.5cm]
Midterm 3}
\author{Josh Fernandes\\
Department of Physics \& Astronomy\\
California State University Long Beach}
\date{\today}

  
\maketitle



\begin{abstract}
This paper examines two different questions. 
\end{abstract}


\section{Problem 1}

\section{The Fortran95 code}

Numtype is the same for problems 1 and 2. 
\begin{lstlisting}[frame=single,caption={Module {\tt NumType}},label=module]

module NumType

	save
	integer, parameter :: 	dp = kind(1.d0)	
	real(dp), parameter :: 	pi = 4*atan(1._dp)
	complex(dp), parameter :: 	iic = (0._dp,1._dp),&	
								one = (1._dp,0._dp),&
								zero = (0._dp,0._dp)


end module NumType

\end{lstlisting}

\begin{lstlisting}[frame=single,caption={ {\tt newton.f95}},label=module]

!this program solves a set of linear equations
module setup

	use numtype
	use matrix
	implicit none
	integer, parameter 	:: nv = 3, np = 3
	real(dp) 			:: a(1:np)

end module setup


program newton

	use setup
	implicit none

	real(dp) 			:: x(nv), dx(nv), f(nv), jacobi(nv,nv), ff
	real(dp), parameter :: eps = 1.e-10_dp
	integer				:: maxstep, i

	a(1:np) = (/ 0.2_dp, 0.2_dp, -100._dp /) !(a,b,c)
	x(1:nv) = (/ -1._dp, 0._dp, 0._dp /) 	!(x,y,z)

	maxstep = 15

	do i = 1, maxstep

		call func(nv,x,f,jacobi,ff)
		print '(3f12.4, 3x, 3e12.4, 3x, e12.4)', &
				x(1:nv), f(1:nv), ff

		if (ff <= eps ) exit

		call dgei(nv,jacobi,nv) 

		dx(1:nv) = matmul( jacobi(1:nv,1:nv), -f(1:nv))

		x(1:nv) = x(1:nv)+dx(1:nv)

	end do


end program newton

subroutine func(n,x,f,jmat,ff)

	use setup
	implicit none
	real(dp) :: x(n), f(n), jmat(n,n), ff
	integer :: n

	f(1) = -(x(2)+x(3))
	f(2) = x(1)+a(1)*x(2)
	f(3) = a(2)+x(3)*(x(1)-a(3))

	ff = sqrt(dot_product(f(1:n),f(1:n)))

	jmat(1,1) = 0
	jmat(2,1) = 1
	jmat(3,1) = x(3)

	jmat(1,2) = -1
	jmat(2,2) = a(1)
	jmat(3,2) = 0

	jmat(1,3) = -1
	jmat(2,3) = 0
	jmat(3,3) = x(1)-a(3)

end subroutine func

\end{lstlisting}

The main program is {\tt newton} and it begins with its own module. 


The code is run by typing {\tt ./newton}.

\section{Problem 2}



\section{The Fortran95 code}

\begin{lstlisting}[frame=single,caption={ {\tt mtestthisone.f95}},label=module]

module setup

	use NumType
	implicit none
	integer,	parameter	::	n_basis=10,lwork=2*n_basis+1
	real(dp),	parameter	::	mass=1.0_dp, hbar=1.0_dp,&
							  	omega_h=2._dp,omega_b=1._dp

end module setup

program matrix

	use setup
	implicit none

	complex(dp) ::	x_mat(0:n_basis,0:n_basis+1), p_mat(0:n_basis,0:n_basis+1), &
					x2_mat(0:n_basis,0:n_basis), p2_mat(0:n_basis,0:n_basis), &
					h_mat(0:n_basis,0:n_basis),f(0:n_basis,0:n_basis), &
                    work(lwork),c(0:n_basis,0:n_basis),d(0:n_basis,0:n_basis)
    integer :: n,m, info, i, j,ipiv(n_basis)
    real(dp) :: rwork(3*(n_basis-2)), w_eigen(n_basis+1)


	x_mat = 0._dp
    p_mat = 0._dp

	do n=0,n_basis-1
		x_mat(n,n+1) = sqrt(hbar/(2*mass*omega_b))*sqrt(n+1._dp)
		x_mat(n+1,n) = x_mat(n,n+1)		
	end do
	x_mat(n_basis,n_basis+1) = sqrt(hbar/(2*mass*omega_b))*sqrt(n_basis+1._dp) !add last point


    do n=0,n_basis-1
        p_mat(n,n+1) = -iic*sqrt(mass*hbar*omega_b/2)*sqrt(n+1._dp)
        p_mat(n+1,n) = -p_mat(n,n+1)     
    end do
    p_mat(n_basis,n_basis+1) = -sqrt(mass*hbar*omega_b/2)*sqrt(n_basis+1._dp) !add last point


   ! do n=0,n_basis
!        print '(12f10.5)', x_mat(n,0:n_basis+1)
!    end do
!
!    print *, '----------'
!
!    do n=0,n_basis
!        print '(12f10.5)', p_mat(n,0:n_basis+1)
!    end do


     print *, '==========================================================='


    h_mat(0:n_basis,0:n_basis) = 1/(2*mass) * &
        matmul(p_mat(0:n_basis,0:n_basis+1),conjg(transpose(p_mat(0:n_basis,0:n_basis+1)))) + &
        mass*omega_h**2/2 * &
        matmul(x_mat(0:n_basis,0:n_basis+1),transpose(x_mat(0:n_basis,0:n_basis+1)))
        
    h_mat(0,5) = h_mat(0,5) + (hbar*omega_h/2) !add contribution
    h_mat(5,0) = h_mat(5,0) + (hbar*omega_h/2)
    
    print *, '---------------hamiltonian matrix---------------'
    do n=0,n_basis
        print '(12f10.5)', h_mat(n,0:n_basis)
    end do

    call zheev('n','u',n_basis+1,h_mat,n_basis+1,w_eigen,work,lwork,rwork,info)


    print *, '----------'
    print *, info

    print *, 'eigenvalue', w_eigen(1:n_basis+1)
    
    do i = 1,10
		print *,'  vector  ', dble(h_mat(0:n_basis,i)) 
	end do
	
	!do n=1,n_basis
!        print '(12f10.5)', h_mat(1:n_basis,n)
!    end do
    
    
    print *, '------- Orthogonality ------- '
    
 	do j = 1,10
            do i = 1,10
                print *, i, j, dot_product(h_mat(1:10,i) ,h_mat(1:10,j) ) !determines if matrix is orthogonal
            
            end do
        end do

        
        print *, '------- completeness & spectral decomp------- '
        
        do j = 1,n_basis
            do i = 1,n_basis
                f(i,j)= dot_product(h_mat(i,1:n_basis) ,h_mat(j,1:n_basis)&
                *w_eigen(1:n_basis))  !determines if matrix is complete
            end do
        end do 
        
 
        Do i= 1,10
            print '(10f7.2)', f(i,1:10)
        end do 
        
        
        print *, '+++++++++++++++++++++++++++++++++++'
        
        c(1:n_basis,1:n_basis) = conjg(transpose(h_mat(1:n_basis,1:n_basis)))
        
        f(1:n_basis,1:n_basis) = matmul(c(1:n_basis,1:n_basis),h_mat(1:n_basis,1:n_basis))
        f(1:n_basis,1:n_basis) = matmul(h_mat(1:n_basis,1:n_basis),c(1:n_basis,1:n_basis))
        
        Do i= 1,n_basis
            print '(10f7.2)', f(i,1:n_basis)
        end do 
        
        print *, '++++++++++++++++++++++++++++++++++++++++++++'
        
        f(1:n_basis,1:n_basis)=matmul(h_mat(1:n_basis,1:n_basis),h_mat(1:n_basis,1:n_basis))
        d(1:n_basis,1:n_basis) = matmul(c(1:n_basis,1:n_basis),f(1:n_basis,1:n_basis))
        
        Do i= 1,n_basis
            print '(10f7.2)', d(i,1:n_basis)
        end do 
        
        
        print *, '++++++++ inversions of e++++++++++++++ '
        info = 0
        call zgetrf(n_basis,n_basis,h_mat, n_basis, ipiv,info)
        
        if(info/= 0) stop 'stop'
        
        call zgetri(n_basis,h_mat,n_basis,ipiv,work, lwork,info)
        
        if(info/= 0) stop 'stop'
        
        Do i= 1,n_basis
            print '(6f7.2,5x,6f7.2)', c(i,1:n_basis)-h_mat(i,1:n_basis)
        end do 
            
 
     

end program matrix

\end{lstlisting}



\section{Results}

Here are the results for the first problem. This is for part a. It would seem that changing the value of c significantly changes the resulting values of x,y, and z. When $a=.2,b=.2,c=5.7$, then $x=.0070,y=-0.0351,z=0.0351$. When $a=.2,b=.2,c=5.6$, then $x=.0072,y=-0.0358,z=0.0358$. When $a=.2,b=.2,c=5.8$, then $x=.0069,y=-0.0345,z=0.0345$. When $a=.2,b=.2,c=5$, then $x=.0080,y=-0.0401,z=0.0401$. When $a=.2,b=.2,c=100$, then $x=.0004,y=-0.02,z=0.02$. Making C negative just changes the sign of x,y, and z. For part b, there were only two solutions. When $x=..0070,y=-0.0351,z=0.0351$ and $x=5.6930,y=-28.4649,z=28.4649,$. It didn't matter how far away the initial conditions were, they would always settle to one of the two conditions. In contrast, changing c almost always resulted in a differenent solution for x,y,z.

for the second problem, I should be getting a matrix with 1's on the diagonal to prove that it is orthogonal. That is not what I got.



\begin{thebibliography}{}


\bibitem{metcalf} M.\ Metcalf, J.\ Reid and M.\ Cohen, {\it Fortran 95/2003 explained}. Oxford University Press, 2004.
 

\end{thebibliography}




\end{document}
