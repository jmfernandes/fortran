

\documentclass[12pt]{article}
\usepackage{graphicx}
\usepackage{amsmath}
\usepackage{listings}
\usepackage{color}
\usepackage{braket}
\usepackage[section]{placeins} %this stops the figures from showing up in wrong section

\definecolor{dkgreen}{rgb}{0,0.6,0}
\definecolor{dkblue}{rgb}{0,0.0,0.6}
\definecolor{dkred}{rgb}{0.9,0.0,0.1}


\begin{document}

\lstset{language=Fortran,tabsize=4,numbers=left,numberstyle=\tiny,basicstyle=\ttfamily\small\color{dkblue},stringstyle=\ttfamily\color{blue},keywordstyle=\rmfamily\color{dkred}\bfseries\emph,backgroundcolor=\color{white},commentstyle=\color{dkgreen}}




\title{Physics 562 - Computational Physics\\[.5cm]
Assignment 4: Eigenvalues of Simple Harmonic Oscillator}
\author{Josh Fernandes\\
Department of Physics \& Astronomy\\
California State University Long Beach}
\date{\today}

  
\maketitle



\begin{abstract}
This paper examines the eigenvalues and eigenvectors of two one-dimensional harmonic oscillators. 
\end{abstract}

\section{Hamiltonian}
one dimensional harmonic oscillator is given by the differential operator
\begin{gather}
m \frac{\partial^2 x}{\partial t^2} = -kx.
\end{gather}
This differential equation has the solution
\begin{gather}
x=x_0 sin(wt + \theta),
\end{gather}
where
\begin{gather}
w=\sqrt{\frac{k}{m}}.
\end{gather}
The total energy of the system is the same as the hamiltonian which is given by
\begin{gather}
E = \mathcal{H} = \frac{p^2}{2m} + \frac{mw^2x^2}{2}
\end{gather}
Now, make the following substitutions of the form 
\begin{gather}
\xi = x \sqrt{\frac{mw}{\hbar}}, \\
\pi = \frac{p}{\sqrt{\hbar mw}}. 
\end{gather}
This gives the equation
\begin{gather}
\mathcal{H} = \frac{\hbar w}{2}(\pi^2+\xi^2).
\end{gather}
The expression can be factorized so that it then becomes
\begin{gather}
\mathcal{H} = \frac{\hbar w}{2}[(\xi + i \pi)(\xi - i \pi) + (\xi - i \pi)(\xi + i \pi)].
\end{gather}
The following two operators can be defined in terms of the position and momentum, 
\begin{gather}
a = \frac{\xi + i \pi}{\sqrt{2}} = \frac{1}{\sqrt{2 \hbar m w}}(mwx + ip) \\
a^\dagger = \frac{\xi - i \pi}{\sqrt{2}} = \frac{1}{\sqrt{2 \hbar m w}}(mwx - ip).
\end{gather}
from the commutator relation $[i \pi,\xi] = 1$, it follows that $[a,a^\dagger] = 1$. Using this, the hamiltonian can be written in the form
\begin{gather}
\mathcal{H} = \hbar w (a^\dagger a + \frac{1}{2}) = \hbar w (N + \frac{1}{2}).
\label{hamil}
\end{gather}
Note that $N = a^\dagger a$. The energy eigenvalues is given by
\begin{gather}
\mathcal{H}\Ket{n} = (n+\frac{1}{2})\hbar w\Ket{n}
\end{gather}
The only nonzero matrix elements of $a^\dagger$ are $\Bra{n+1} a^\dagger \Ket{n} = \sqrt{n+1}$. So, this gives the matrix
\[ a^\dagger = \left( \begin{array}{ccccc}
0 & 0 & 0 & 0 & \ldots \\
\sqrt{1} & 0 & 0 & 0 & \ldots \\
0 & \sqrt{2} & 0 & 0 & \ldots \\
0 & 0 & \sqrt{3} & 0 & \ldots \\
\vdots & \vdots & \vdots & \vdots & \ddots \end{array} \right),\]
and the adjoint
\[ a= \left( \begin{array}{ccccc}
0 & \sqrt{1}  & 0 & 0 & \ldots \\
0 & 0 & \sqrt{2}  & 0 & \ldots \\
0 & 0 & 0 & \sqrt{3}  & \ldots \\
0 & 0 & 0 & 0 & \ldots \\
\vdots & \vdots & \vdots & \vdots & \ddots \end{array} \right).\]
Now, the position and momentum can be defined in terms of $a$ and $a^\dagger$. This gives $x$ and $p$ in the form
\begin{gather}
x = \sqrt{\frac{\hbar}{2mw}}(a^\dagger + a) \\
p = i\sqrt{\frac{mw\hbar}{2}}(a^\dagger - a).
\end{gather}
Plugging in for $a^\dagger$ and $a$ results in
\[ x= \sqrt{\frac{\hbar}{2mw}} \left( \begin{array}{ccccc}
0 & \sqrt{1}  & 0 & 0 & \ldots \\
\sqrt{1} & 0 & \sqrt{2}  & 0 & \ldots \\
0 & \sqrt{2} & 0 & \sqrt{3}  & \ldots \\
0 & 0 & \sqrt{3} & 0 & \ldots \\
\vdots & \vdots & \vdots & \vdots & \ddots \end{array} \right)\]
and
\[ p= i\sqrt{\frac{mw\hbar}{2}} \left( \begin{array}{ccccc}
0 & -\sqrt{1}  & 0 & 0 & \ldots \\
\sqrt{1} & 0 & -\sqrt{2}  & 0 & \ldots \\
0 & \sqrt{2} & 0 & -\sqrt{3}  & \ldots \\
0 & 0 & \sqrt{3} & 0 & \ldots \\
\vdots & \vdots & \vdots & \vdots & \ddots \end{array} \right)\]

Now look at the hamiltonians of two different harmonic oscillators. The first harmonic oscillator has the parameters of $\hbar = 1$, $m = 1$, and $w=.5$. The second harmonic oscillator has the parameters of $\hbar = 1$, $m = 1$, and $w=1$. If you calculate the hamiltonians using equation \ref{hamil} you get
\begin{gather}
\mathcal{H}_1 =  (N + \frac{1}{2}) \\
\mathcal{H}_2 =  .5 \cdot(N + \frac{1}{2})
\end{gather}
writing as a $10\times10$ matrix we get
\[ \mathcal{H}_1 = \left( \begin{array}{cccccccccc}
\frac{1}{2} & 0 & 0 & 0 & 0 & 0 & 0 & 0 & 0 & 0\\
0 & \frac{3}{2} & 0 & 0 & 0 & 0 & 0 & 0 & 0 & 0\\
0 & 0 & \frac{5}{2} & 0 & 0 & 0 & 0 & 0 & 0 & 0\\
0 & 0 & 0 & \frac{7}{2} & 0 & 0 & 0 & 0 & 0 & 0\\
0 & 0 & 0 & 0 & \frac{9}{2} & 0 & 0 & 0 & 0 & 0\\
0 & 0 & 0 & 0 & 0 & \frac{11}{2} & 0 & 0 & 0 & 0\\
0 & 0 & 0 & 0 & 0 & 0 & \frac{13}{2} & 0 & 0 & 0\\
0 & 0 & 0 & 0 & 0 & 0 & 0 & \frac{15}{2} & 0 & 0\\
0 & 0 & 0 & 0 & 0 & 0 & 0 & 0 & \frac{17}{2} & 0\\
0 & 0 & 0 & 0 & 0 & 0 & 0 & 0 & 0 & \frac{19}{2} \end{array} \right)\]
and
\[ \mathcal{H}_2 = \left( \begin{array}{cccccccccc}
\frac{1}{4} & 0 & 0 & 0 & 0 & 0 & 0 & 0 & 0 & 0\\
0 & \frac{3}{4} & 0 & 0 & 0 & 0 & 0 & 0 & 0 & 0\\
0 & 0 & \frac{5}{4} & 0 & 0 & 0 & 0 & 0 & 0 & 0\\
0 & 0 & 0 & \frac{7}{4} & 0 & 0 & 0 & 0 & 0 & 0\\
0 & 0 & 0 & 0 & \frac{9}{4} & 0 & 0 & 0 & 0 & 0\\
0 & 0 & 0 & 0 & 0 & \frac{11}{4} & 0 & 0 & 0 & 0\\
0 & 0 & 0 & 0 & 0 & 0 & \frac{13}{4} & 0 & 0 & 0\\
0 & 0 & 0 & 0 & 0 & 0 & 0 & \frac{15}{4} & 0 & 0\\
0 & 0 & 0 & 0 & 0 & 0 & 0 & 0 & \frac{17}{4} & 0\\
0 & 0 & 0 & 0 & 0 & 0 & 0 & 0 & 0 & \frac{19}{4} \end{array} \right)\]
In the system that we are looking at, the first harmonic oscillator is going to be in the basis of the second harmonic oscillator. The new hamiltonian will be of the form 
\begin{gather}
\mathcal{H} =   \mathcal{H}_1 + \mathcal{H}_2 + \frac{1}{2} m \cdot (\omega_2-\omega_1)^2 \cdot (x_2-x_1)^2
\label{hamil2}
\end{gather}
where $w_2$ and $x_2$ are the matrices associated with the second harmonic oscillator, and $w_1$ and $x_1$ are the matrices associated with the first harmonic oscillator. The eigenvalues and eigenvector will now be calculated for the hamiltonian given in equation \ref{hamil2}.

\section{The Fortran95 code}

The code calculates the eigenvalues and eigenvectors of a matrix of any size. For this project, a $10 \times 10$ matrix is used. First a module called {\tt NumType} is created to store all my global parameters.

\begin{lstlisting}[frame=single,caption={Module {\tt NumType}},label=module]

module NumType

	save
	integer, parameter		::	dp = kind(1.d0)
	real(dp), parameter		::	pi = 4*atan(1._dp)
	complex(dp), parameter	:: 	iic = (0._dp,1._dp),&
								one = (1._dp,0._dp),&
								zero = (0._dp,0._dp)

end module NumType
\end{lstlisting}

\begin{lstlisting}[frame=single,caption={ {\tt mtest.f95}},label=module]

module setup

	use NumType
	implicit none
	integer,	parameter	::  ndim=10,lwork=5*ndim
	real(dp),	parameter	::  mass=1.0_dp, &
				        	    hbar=1.0_dp,&
					 		    omega1=0.5_dp,& 
							    omega2=1._dp
	
end module setup

program matrix

	use setup
	implicit none

	complex(dp), dimension(ndim,ndim)	:: A,B,E,H1,H2,H3
	real(dp),	 dimension(ndim)		:: w
	integer 							:: i, nn, info
	complex(dp) 						:: work(lwork)
	real(dp) 							:: rwork(lwork)

	A(1:10,1:10) = reshape((/	             		    &
			zero, sqrt(1*one), zero, zero, zero,		&
			zero, zero, zero, zero, zero,				&
			sqrt(1*one), zero, sqrt(2*one), zero, zero,	&
			zero, zero, zero, zero, zero,				&
			zero, sqrt(2*one), zero, sqrt(3*one), zero,	&
			zero, zero, zero, zero, zero,				&
			zero, zero, sqrt(3*one), zero, sqrt(4*one), &
			zero, zero, zero, zero, zero,				&
			zero, zero, zero, sqrt(4*one), zero,		&
			sqrt(5*one), zero, zero, zero, zero,		&
			zero, zero, zero, zero, sqrt(5*one),		&
			zero, sqrt(6*one), zero, zero, zero,		&
			zero, zero, zero, zero, zero,				&
			sqrt(6*one), zero, sqrt(7*one), zero, zero,	&
			zero, zero, zero, zero, zero,				&
			zero, sqrt(7*one), zero, sqrt(8*one), zero,	&
			zero, zero, zero, zero, zero,				&
			zero, zero, sqrt(8*one), zero, sqrt(9*one),	&
			zero, zero, zero, zero, zero,				&
			zero, zero, zero, sqrt(9*one), zero 		&
	/), 												&
	(/10,10/))

	H1(1:10,1:10) = reshape((/  						&
			1/2._dp*one, zero, zero, zero, zero,		&
			zero, zero, zero, zero, zero,				&
			zero, 3/2._dp*one, zero, zero, zero,		&
			zero, zero, zero, zero, zero,				&
			zero, zero, 5/2._dp*one, zero, zero,		&
			zero, zero, zero, zero, zero,				&
			zero, zero, zero, 7/2._dp*one, zero,		&
			zero, zero, zero, zero, zero,				&
			zero, zero, zero, zero, 9/2._dp*one,		&
			zero, zero, zero, zero, zero,				&
			zero, zero, zero, zero, zero,				&
			11/2._dp*one, zero, zero, zero, zero,		&
			zero, zero, zero, zero, zero,				&
			zero, 13/2._dp*one, zero, zero, zero,		&
			zero, zero, zero, zero, zero,				&
			zero, zero, 15/2._dp*one, zero, zero,		&
			zero, zero, zero, zero, zero,				&
			zero, zero, zero, 17/2._dp*one, zero,		&
			zero, zero, zero, zero, zero,				&
			zero, zero, zero, zero, 19/2._dp*one 		&
	/), 												&
	(/10,10/))

	H2(1:10,1:10) = reshape((/							&
			1/4._dp*one, zero, zero, zero, zero,		&
			zero, zero, zero, zero, zero,				&
			zero, 3/4._dp*one, zero, zero, zero,		&
			zero, zero, zero, zero, zero,				&
			zero, zero, 5/4._dp*one, zero, zero,		&
			zero, zero, zero, zero, zero,				&
			zero, zero, zero, 7/4._dp*one, zero,		&
			zero, zero, zero, zero, zero,				&
			zero, zero, zero, zero, 9/4._dp*one,		&
			zero, zero, zero, zero, zero,				&
			zero, zero, zero, zero, zero,				&
			11/4._dp*one, zero, zero, zero, zero,		&
			zero, zero, zero, zero, zero,				&
			zero, 13/4._dp*one, zero, zero, zero,		&
			zero, zero, zero, zero, zero,				&
			zero, zero, 15/4._dp*one, zero, zero,		&
			zero, zero, zero, zero, zero,				&
			zero, zero, zero, 17/4._dp*one, zero,		&
			zero, zero, zero, zero, zero,				&
			zero, zero, zero, zero, 19/4._dp*one 		&
	/), 												&
	(/10,10/))

	A(1:10,1:10)= sqrt(hbar/(2*mass*omega1))*A(1:10,1:10)
	B(1:10,1:10)= sqrt(hbar/(2*mass*omega2))*A(1:10,1:10)
	H3(1:10,1:10)= H1 + H2 + 1/2._dp*mass*&
				   (omega2-omega1)**2*(B-A)**2

	nn = 10
	info = 0
	E(1:nn,1:nn) = H3(1:nn,1:nn)

	call zheev('v','u',nn,E,ndim,w,work,lwork,rwork,info)

	print *, 'info=', info

	do i = 1,10
		print '(a,f15.8,a,20f6.0)','eigenvalues',w(i),&
		'vector', dble(e(1:nn,i)) 
	end do

end program matrix

\end{lstlisting}



The code is run by typing {\tt ./mat}. The resulting eigenvalues and eigenvectors are printed out to the terminal.

\section{Results}

\begin{table}[h]
\centering
\begin{tabular}{cc}
\hline
             & eigenvalue \\ \hline\hline  
ground state & {$\frac{3}{4}$}         \\ [1ex]
1st excited  & {$\frac{9}{4}$}        \\ [1ex]
2nd excited  & {$\frac{15}{4}$}         \\ [1ex]
3rd excited  & {$\frac{21}{4}$}         \\ [1ex]
4th excited  & {$\frac{27}{4}$}        \\ [1ex]
5th excited  & {$\frac{33}{4}$}         \\ [1ex]
6th excited  & {$\frac{39}{4}$}         \\ [1ex]
7th excited  & {$\frac{45}{4}$}        \\ [1ex]
8th excited  & {$\frac{51}{4}$}        \\ [1ex]
9th excited  & {$\frac{57}{4}$}       \\  \hline
\end{tabular}
\end{table}

The eigenvalues are not the same as the eigenvalues of the individual harmonic oscillators. This would indicate that the harmonic oscillators are coupled.  


\section{Summary and conclusions}

The eigenvalues tell us that Natalie is super cool. 

\begin{thebibliography}{}


\bibitem{metcalf} M.\ Metcalf, J.\ Reid and M.\ Cohen, {\it Fortran 95/2003 explained}. Oxford University Press, 2004.
 

\end{thebibliography}




\end{document}
