

\documentclass[12pt]{article}
\usepackage{graphicx}
\usepackage{amsmath}
\usepackage{listings}
\usepackage{color}
\usepackage[section]{placeins} %this stops the figures from showing up in wrong section

\definecolor{dkgreen}{rgb}{0,0.6,0}
\definecolor{dkblue}{rgb}{0,0.0,0.6}
\definecolor{dkred}{rgb}{0.9,0.0,0.1}


\begin{document}

\lstset{language=Fortran,tabsize=4,numbers=left,numberstyle=\tiny,basicstyle=\ttfamily\small\color{dkblue},stringstyle=\ttfamily\color{blue},keywordstyle=\rmfamily\color{dkred}\bfseries\emph,backgroundcolor=\color{white},commentstyle=\color{dkgreen}}




\title{Physics 562 - Computational Physics\\[.5cm]
Midterm 4}
\author{Josh Fernandes\\
Department of Physics \& Astronomy\\
California State University Long Beach}
\date{\today}

  
\maketitle



\begin{abstract}
This paper examines two different questions. 
\end{abstract}


\section{Problem 1}

\section{The Fortran95 code}

Numtype is the same for problems 1 and 2. 
\begin{lstlisting}[frame=single,caption={Module {\tt NumType}},label=module]


\end{lstlisting}

The main program is {\tt newton} and it begins with its own module. 


The code is run by typing {\tt ./newton}.

\section{Problem 2}



\section{The Fortran95 code}

\begin{lstlisting}[frame=single,caption={ {\tt mtestthisone.f95}},label=module]


\end{lstlisting}



\section{Results}





\begin{thebibliography}{}


\bibitem{metcalf} M.\ Metcalf, J.\ Reid and M.\ Cohen, {\it Fortran 95/2003 explained}. Oxford University Press, 2004.
 

\end{thebibliography}




\end{document}
